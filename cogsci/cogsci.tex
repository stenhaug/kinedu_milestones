% Template for Cogsci submission with R Markdown

% Stuff changed from original Markdown PLOS Template
\documentclass[10pt, letterpaper]{article}

\usepackage{cogsci}
\usepackage{pslatex}
\usepackage{float}
\usepackage{caption}

% amsmath package, useful for mathematical formulas
\usepackage{amsmath}

% amssymb package, useful for mathematical symbols
\usepackage{amssymb}

% hyperref package, useful for hyperlinks
\usepackage{hyperref}

% graphicx package, useful for including eps and pdf graphics
% include graphics with the command \includegraphics
\usepackage{graphicx}

% Sweave(-like)
\usepackage{fancyvrb}
\DefineVerbatimEnvironment{Sinput}{Verbatim}{fontshape=sl}
\DefineVerbatimEnvironment{Soutput}{Verbatim}{}
\DefineVerbatimEnvironment{Scode}{Verbatim}{fontshape=sl}
\newenvironment{Schunk}{}{}
\DefineVerbatimEnvironment{Code}{Verbatim}{}
\DefineVerbatimEnvironment{CodeInput}{Verbatim}{fontshape=sl}
\DefineVerbatimEnvironment{CodeOutput}{Verbatim}{}
\newenvironment{CodeChunk}{}{}

% cite package, to clean up citations in the main text. Do not remove.
\usepackage{apacite}

% KM added 1/4/18 to allow control of blind submission
\cogscifinalcopy

\usepackage{color}

% Use doublespacing - comment out for single spacing
%\usepackage{setspace}
%\doublespacing


% % Text layout
% \topmargin 0.0cm
% \oddsidemargin 0.5cm
% \evensidemargin 0.5cm
% \textwidth 16cm
% \textheight 21cm

\title{The latent factor structure of child development}

\usepackage{bm}

\author{Anonymous Cogsci Submission}

\begin{document}

\maketitle

\begin{abstract}
Hello

\textbf{Keywords:}
one; two;
\end{abstract}

\hypertarget{introduction}{%
\section{Introduction}\label{introduction}}

A child's development can be thought of as the set of developmental
milestones that they have reached at a particular point in time. This
conceptualization results in data with the same structure as the item
response data common to educational measurement. In education, item
response data is most typically students responding to test items (i.e.,
questions) and, in the dichotomous case, getting each question either
correct or incorrect. In the context of child development, the child is
the ``student,'' and each developmental milestone is the ``item.''

We use Kinedu, a Mexico-based child development app, as a source for
this type of data. When parents first start using the Kinedu app, they
are asked a series of questions about which developmental milestones
their child has reached. We consider the 1946 children between 2 and 55
months of age whose parents responded to all 414 of the developmental
milestones. Each developmental miletone on Kinedu is mapped to a
milestone group: physical, cognitive, linguistic, or social \&
emotional. Table 1 shows the number of developmental milestones in each
group along with an example milestone in Spanish (as its shown to the
parent) and its translation in English.

Figure 1 shows the age (in months) and number of developmental
milestones for each child. As can be seen in Figure 1, at 12 months of
age, most children have reached about 200 developmental milestones. At
24 months of age, most children have reached about 300 developmental
milestones. Finally, at 48 months of age, most children have reached
about 375 of the 414 developmental milestones.

\hypertarget{empirical-assessment-of-the-dimensionality-of-child-development}{%
\section{Empirical assessment of the dimensionality of child
development}\label{empirical-assessment-of-the-dimensionality-of-child-development}}

We frame the assessment of the dimensionality of child development as a
model comparison question.

\hypertarget{models}{%
\subsection{Models}\label{models}}

Item response theory offers a suite of models with which to model item
response data. We adopt the notation used in Chalmers \& others (2012).
Let \(i = 1, \ldots, I\) represent the distinct children and
\(j = 1, \ldots, J\) the developmental milestones. The Kinedu item
response data is stored in a matrix, \(y\), where element \(y_{ij}\)
denotes if the \(i\)th child has or has not achieved the \(j\)th
developmental milestone as reported by their parent/guardian. Each model
represents the \(i\)th child's development using \(m\) latent factors
\(\boldsymbol{\theta}_{i}=(\theta_1, \ldots, \theta_m)\). The \(j\)th
milestone's discriminations (i.e.~slopes)
\(\boldsymbol{\alpha}_{j}=(\alpha_{1}, \dots, \alpha_{m})\) capture the
latent factor loadings onto that milestone.

\hypertarget{acknowledgements}{%
\section{Acknowledgements}\label{acknowledgements}}

We'd like to thank Kinedu for providing the data that made this research
possible.

\hypertarget{references}{%
\section{References}\label{references}}

\setlength{\parindent}{-0.1in} 
\setlength{\leftskip}{0.125in}

\noindent

\hypertarget{refs}{}
\leavevmode\hypertarget{ref-chalmers2012mirt}{}%
Chalmers, R. P., \& others. (2012). Mirt: A multidimensional item
response theory package for the r environment. \emph{Journal of
Statistical Software}, \emph{48}(6), 1--29.

\bibliographystyle{apacite}


\end{document}
